%%%%%%%%%%%%%%%%%%%%%%%%%%%%%%%%%%%%%%%%%
% fphw Assignment
% LaTeX Template
% Version 1.0 (27/04/2019)
%
% This template originates from:
% https://www.LaTeXTemplates.com
%
% Authors:
% Class by Felipe Portales-Oliva (f.portales.oliva@gmail.com) with template 
% content and modifications by Vel (vel@LaTeXTemplates.com)
%
% Template (this file) License:
% CC BY-NC-SA 3.0 (http://creativecommons.org/licenses/by-nc-sa/3.0/)
%
%%%%%%%%%%%%%%%%%%%%%%%%%%%%%%%%%%%%%%%%%

%----------------------------------------------------------------------------------------
%	PACKAGES AND OTHER DOCUMENT CONFIGURATIONS
%----------------------------------------------------------------------------------------

\documentclass[
	12pt, % Default font size, values between 10pt-12pt are allowed
	%letterpaper, % Uncomment for US letter paper size
	%spanish, % Uncomment for Spanish
]{fphw}

% Template-specific packages
\usepackage[utf8]{inputenc} % Required for inputting international characters
\usepackage[T1]{fontenc} % Output font encoding for international characters
\usepackage{mathpazo} % Use the Palatino font

\usepackage{graphicx} % Required for including images

\usepackage{booktabs} % Required for better horizontal rules in tables

\usepackage{listings} % Required for insertion of code

\usepackage{enumerate} % To modify the enumerate environment

%----------------------------------------------------------------------------------------
%	ASSIGNMENT INFORMATION
%----------------------------------------------------------------------------------------

\title{Homework \#2} % Assignment title

\author{Yitian Wang yw3378} % Student name

\date{\today} % Due date

\institute{Columbia University} %Institute or school name

\class{MSAE-E4215 Mechanical behavior of materials} % Course or class name

\professor{Prof. William Bailey} % Professor or teacher in charge of the assignment

%----------------------------------------------------------------------------------------

\begin{document}

\maketitle % Output the assignment title, created automatically using the information in the custom commands above

%----------------------------------------------------------------------------------------
%	ASSIGNMENT CONTENT
%----------------------------------------------------------------------------------------

\section*{Question 1.2}

\begin{problem}
Determine the polarization character of the optical and acoustic modes
of the diatomic lattice by backsubstituting the eigenfrequencies ω in
Eq 1.39 into the eigenvalue Equation 1.34, and solving for A and B.
How do the atoms move?
\end{problem}

%------------------------------------------------

\subsection*{Answer}



%----------------------------------------------------------------------------------------

\section*{Question 1.3}

\begin{problem}
Derive the dispersion relation for a monatomic lattice: A atoms only,
separated by a lattice spacing of a, with spring constant k. Using the
Bloch theorem, show (graphically) that the result is identical to that
for the diatomic lattice with $m_A = m_B$ shown in Figure 1.3. Hint: the
first Brillouin zone boundary for a lattice with lattice parameter 2a
is at half the value of that for a lattice with lattice parameter a, and
the reduced representation for one is in the extended representation
for the other.

\end{problem}

%------------------------------------------------

\subsection*{Answer}



%----------------------------------------------------------------------------------------

\section*{Question 1.5}

\begin{problem}
Calculate the Debye temperature for a fictitious FCC crystal with 100
g/mol, lattice parameter of 0.4 nm, and isotropic speed of sound 5
km/s.
\end{problem}

%------------------------------------------------

\subsection*{Answer} 



%----------------------------------------------------------------------------------------

\section*{Question 1.6}

\begin{problem}
Derive the low-temperature limit of heat capacity, $C_V (T)$, for a twodimensional lattice, in terms of the Debye temperature $\theta$ and the Boltzmann constant $kB$ only.
\end{problem}

%------------------------------------------------

\subsection*{Answer}



%----------------------------------------------------------------------------------------

\section*{Question 2.1}

\begin{problem}
Consider a solid cylinder, radius R, length l, density ρ standing up
on its end, so that its total height (in x3) above the pavement is l.
Assume normal stresses only. Derive an expression for the stress in
the material as a function of x3, and plot σ33(x3). How tall can a solid
column of concrete (ρ = 2 g/cm3
) be made before the maximum shear
stress exceeds the critical shear stress for fracture, τc = 10 MPa? Take
σ33 ∼ τmax at fracture. (Recall that the total forces exerted on the
body have to sum to zero.)
\end{problem}

%------------------------------------------------

\subsection*{Answer}

\begin{enumerate}

\end{enumerate}

%----------------------------------------------------------------------------------------

\section*{Question 2.2}

\begin{problem}
Consider a small sphere of NdFeB permanent magnet material in a
large, uniform, superconducting magnetic field of 10 T. If its magnetization M is initially orthogonal to the applied field µ0H when the
field is (instantaneously) turned on, does the sphere shatter? Take
τc = 50 MPa and µ0Ms = 0.5 T
\end{problem}

%------------------------------------------------

\subsection*{Answer}

\begin{enumerate}

\end{enumerate}

%----------------------------------------------------------------------------------------

\section*{Question 2.3}

\begin{problem}
There is a longstanding proposal for a space elevator, as pictured in
Figure 2.5. If a mass M can be attached to a long cable, length l,
and $l\farlarger R$, where R is the radius of the earth, gravity is less strong
than the centrifugal force, the cable might be supported. 1) Assume
that there is no counterweight (M = 0). For a cable density ρ, how
long does the cable need to be for it to stand up (i.e. all sections
in tension)? Estimate l/R for a density of ρ = 2 g/cm3
, appropriate
for carbon fiber. 2) Plot the stress in the cable, and calculate the maximum stress.
\end{problem}

%------------------------------------------------

\subsection*{Answer}

\begin{enumerate}

\end{enumerate}

%----------------------------------------------------------------------------------------

\section*{Question 3.01}

\begin{problem}
Verify by direct substitution that for the a_{ij} matrix presented for transformation from Cartesian to spherical cooordinates, the reverse transformation from spherical to Cartesian coordinates is the transpose a_{ji}.  (Use the equation presented in class; 3.26 in the notes needs to be fixed.)
\end{problem}

%------------------------------------------------

\subsection*{Answer}

\begin{enumerate}

\end{enumerate}

%----------------------------------------------------------------------------------------

\section*{Question 3.02}

\begin{problem}
Write an expression for the sum of the diagonal elements of the stress matrix using dummy suffix / Einstein notation
\end{problem}

%------------------------------------------------

\subsection*{Answer}

\begin{enumerate}

\end{enumerate}

%----------------------------------------------------------------------------------------

\end{document}

