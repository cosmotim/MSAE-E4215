\documentclass{article}
\usepackage{enumitem}
\title{MSAE E4215 Homework \#1}
\author{Yitian Wang}
\date{Columbia University \\ \today}


\begin{document}
\maketitle
\begin{enumerate}
\item One widely used empirical potential for the energy between two atomswith spacingdis the Morse potential, 
written as
$U(d) =D(e^{-2α(d - d_0)} - 2e^{ - α(d - d_0)})$
Sample values for Cu are$D= 343 meV,α= 1.36 ̊A^{−1}$.  
For a diatomicbond of Cu2,

\begin{enumerate}[label=(\alph*)]
\item Calculate the ”spring constant”f, assuming an interatomic spacing of 0.209 nm.
\begin{equation}
f=\frac{\partial^2 U}{\partial d^2}
apply taylor series expansion on the U near d=d_0
%U(d)=D(-1 + 0 + a^2(d-d_0)^2 + {-a}^3(d-d_0)^3 +  )

\end{equation}
\item Calculate the anharmonic coefficientgand parameters
\item Calculate the coefficient of thermal expansionαfor the bond
\item Estimate the ratio of elastic modulus at room temperature to thatat zero temperature,
E(300K)/E(0K) considering one bond only.For (FCC) Cu, 
with lattice parameter 0.36nm,
\item calculate the cohesive energy. For simplicity, consider only nearest-neighbor interactions (12 in the crystal).

\end{enumerate}


\item Estimate the yield strength for an absolutely perfect single crystal of Fe 
(and compare with the experimental value)

\item How is the yield strength defined, conventionally? 

\item Brittle or ductile failure: what is better for a structural metal in tension, 
and why?

\item Why might you want to deform a metal plastically? 

\item Examine the stress-strain curve of a perfect Fe whisker (in the presentation, or see notes.)  
a) is it linearly elastic?  b) is it elastic? 


\end{enumerate}
\end{document}
