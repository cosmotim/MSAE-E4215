\documentclass{article}
\usepackage{enumitem}
\title{MSAE E4215 Homework \#1}
\author{Yitian Wang\\UNI: yw3378}
\date{Columbia University \\ \today}


\begin{document}
\maketitle
\begin{enumerate}
\item One widely used empirical potential for the energy between two atomswith spacingdis the Morse potential, 
written as
$U(d) =D(e^{-2a(d - d_0)} - 2e^{ - a(d - d_0)})$
Sample values for Cu are $D= 343 meV, a= 1.36 ̊A^{-1}$.  
For a diatomicbond of Cu2,

\begin{enumerate}[label=(\alph*)]
\item Calculate the "spring constant", assuming an interatomic spacing of 0.209 nm.\\
apply taylor series expansion on the U(d) near $d=d_0$, we have:
$$U(d) \simeq U_{0}+\frac{f}{2 !} u^{2}+\frac{g}{3 !} u^{3}+\frac{h}{4 !} u^{4} \cdots$$
$$U(d)=D(-1 + 0 + a^2(d-d_0)^2 + {-a}^3(d-d_0)^3 + \cdots) $$
the spring constant $f$ is:
$$f=\frac{\partial^2 U}{\partial d^2}=2Da^2 = 1268.8256 \ meV\cdot A^{-2}‬$$

\item Calculate the anharmonic coefficient $g$ and parameter $s$
in this case, refering to the expression in the notes:
%$$f=2Da^2$$
$$g=-6Da^3= -5176.808448 \ meV\cdot A^{-3}‬$$
$$s=\frac{g}{2f}=-\frac{3}{2} a = -2.04 \ A^{-1} $$
\item Calculate the coefficient of thermal expansion $a$ for the bond
from the notes, the thermal expansion coefficient is :
$$a= -s \frac{k_B}{d_0 f} $$
$$ k_B=8.6173 \times 10^{-2} meV \cdot K^{-1}$$
$$d_0=0.209\ nm=2.09\ A $$
$$ a= -s \frac{k_B}{d_0 f} =  6.6291 \times 10^{-5}\cdot K^{-1}$$

\item Estimate the ratio of elastic modulus at room temperature to that at zero temperature,
E(300K)/E(0K) considering one bond only. For (FCC) Cu, with lattice parameter 0.36nm.

$$
\frac{E(T)}{E(0)}=1-\frac{2 s^{2} k_{B}}{f} T = 83.0417 \%
$$
\item calculate the cohesive energy. For simplicity, consider only nearest-neighbor interactions (12 in the crystal).
The total cohesive energy is:
$$ U_{total}=\Sigma^{12}_{1} U(d) $$
from the geometry of FCC structure, 
we know that $a=0.36nm=3.60 A $, $d_0=0.209nm=2.09A$
$$ d=\frac{\sqrt{2}}{2} \cdot a = 2.5456 A $$
$$U(d) =D(e^{-2a(d - d_0)} - 2e^{ - a(d - d_0)}) =-269.84\ meV$$
$$U_{total}=12\times U(d)=-3.2381 eV $$
\end{enumerate}


\item Estimate the yield strength for an absolutely perfect single crystal of Fe 
(and compare with the experimental value)\\
For Fe, the Young's modulus E=210GPa. The estimate yield stress for tension is
$\sigma_y = E/6= 35GPa$. The actual yield stress is 12GPa.

\item How is the yield strength defined, conventionally?\\ 
Yield strength represents the maxium elastic stress or minimum plastic stress. In microscopical view, 
it is the stress that make the largest possible strain under the original structure.
\item Brittle or ductile failure: what is better for a structural metal in tension, and why?\\
Ductile failure is better for a structural metal because it will 
gradually deform the structure instead of collapse in a sudden as the brittle failure, providing
a more predictable progress.

\item Why might you want to deform a metal plastically? \\
Sometime plastic deformation might be a good approach 
to shape the metal into certain product, 
for example, the metal chain production.

\item Examine the stress-strain curve of a perfect Fe whisker (in the presentation, or see notes.) a) is it linearly elastic?  b) is it elastic? \\
It is not linearly elastic for the Young's modulus is not constant with elongation.\\
It is elastic at when the stress us below certain point, 
but is becomes plastic when the stress is high enough.




\end{enumerate}
\end{document}
